\documentclass[a4paper, 12pt]{report}
\usepackage[utf8]{inputenc}
\usepackage[T2A]{fontenc}
\usepackage[english,russian]{babel}
\usepackage[dvips]{graphicx}
\usepackage[left = 2cm, top = 1cm, right = 2cm, bottom = 2cm]{geometry}
\usepackage{hyperref}
\usepackage{xcolor}
\usepackage{titlesec}
\usepackage{listings}
\usepackage{amsthm}
\usepackage{amsmath}
\usepackage{eufrak}

\newtheorem*{definition}{Определение}

\lstloadlanguages{C,[ANSI]C++}%,Clean,make,Fortran}%Загружаемые языки
\lstset{extendedchars=false,
        breaklines=true, %автоперенос длинных линий
        breakatwhitespace=true}
\graphicspath{{pic/}}
\titleformat{\chapter}[block]{\color{black}\Large\bfseries\filcenter}{}{1em}{}
\setcounter{secnumdepth}{0}

\begin{document}
\title{Основания алгебраического подхода к синтезу корректных алгоритмов}
\author{Читал - Рудаков К.В.\\ Набивал - Старожилец В.М.}
\date{}
\maketitle

\tableofcontents

\chapter{Лекция 1}
\section{Введение}
Данные лекции предназначены для общего рассмотрения задач машинного обучения не привязываясь каким-либо конкретным методам. Они являются чем-то сродни взгляда сверху.

В первую очередь следует сформулировать задачу машинного обучения в общем виде. По сути это задача построения такого алгоритма который реализует отображение из множества начальных информаций в множество конечных.

\begin{definition} Символом $\mathfrak{I_i}$ будем обозначать множество начальных информаций. Например, симптомы болезни. \end{definition}
\begin{definition} Символом $\mathfrak{I_f}$ будем обозначать множество конечных информаций. Например, диагноз. \end{definition}

Таким образом, на формальном языке нам требуется найти такой алгоритм $A$, что он осуществляет отображение $\mathfrak{I_i}\xrightarrow{A}\mathfrak{I_f}$. Пока что задача стоит так, что нам нужно просто найти отображение из одного множества в другое причём абсолютно неважно какое. В такой постановке у нас нет каких либо ограничений и даже просто случайный выбор решает эту задачу. Поэтому вводятся дополнительные ограничения на допустимые алгоритмы. Итак,

\begin{definition} $\mathfrak{M}^* = \{A|\ A:\  \mathfrak{I_i}\xrightarrow{A}\mathfrak{I_f}\}$ множество всех алгоритмов осуществляющих отображение из $\mathfrak{I_i}$ в $\mathfrak{I_f}$. \end{definition}
\begin{definition} $I_{str}$ --- структурная информация. Условия/требования по отношению к $A$. \end{definition}
\begin{definition} $\mathfrak{M}(I_{str})$ - некоторое подмножество $\mathfrak{M}^*$ удовлетворяющее $I_{str}$. \end{definition}

Теперь у нас есть некоторый механизм($I_{str}$) позволяющий вводит дополнительные ограничения к нашей задаче.
\begin{definition}[Допустимое отображение]Любое отображение из множества $\mathfrak{M}(I_{str})$ является допустимым. \end{definition}
\begin{definition}[Задача Z] Построение алгоритма реализующего допустимое отображение. \end{definition}
\begin{definition}[Корректный алгоритм] Любой алгоритм реализующий любое допустимое отображение называется корректным. \end{definition}

В такой формулировке очевидно, что необходимое и достаточное условие разрешимости задачи это $\mathfrak{M}(I_{str})\neq\emptyset$, а единственности решения: $|\mathfrak{M}(I_{str})|=1$. 
Заметим также, что в данной формулировке корректный алгоритм - алгоритм не допускающий ни одной ошибки ($\mathfrak{M}(I_{str})$ - множество алгоритмов не допускающих ошибок)! Однако, можно поставить условия и несколько мягче, и дать возможность алгоритмам ошибаться.

\section{Поиск решения задачи}
Корректный алгоритм надо как-то искать, в связи с этим введем ещё одно понятие.

\begin{definition} $\mathfrak{M}(\pi)$ - некоторое параметрическое семейство отображений. \end{definition}

После того, как мы выбрали некоторое $\mathfrak{M}(\pi)$ и впоследствии взяв там какое-нибудь отображение за начальное попытаться попасть в $\mathfrak{M}(I_{str})$. Это возможно если данные семейства пересекаются. Тут нас ждёт дилемма - с одной стороны чем сложнее наше семейство тем выше шанс что оно пересекается с семейством $\mathfrak{M}(I_{str})$, но попасть в это пересечение если $\mathfrak{M}(\pi)$ сложное может быть очень затратно, причём всегда остаётся вероятность, что мы с $\mathfrak{M}(I_{str})$ не пересекаемся. Тут используют идею расширения множества.

\begin{definition} Пусть $f$ - некотороя операция над множеством $\mathfrak{M}^*$. Тогда $f(\mathfrak{M}(\pi))$ расширение множества $\mathfrak{M}(\pi)$. \end{definition}

Таким образом мы стараемся расширить некоторое простое множество до пересечения с $\mathfrak{M}(I_{str})$. Однако, не любая функция $f$ нам подходит. Ведь может получиться, что мы расширились до ''сложного'' множества. Важным является то, что $f$ мы выбираем сами, и можем выбрать его так, чтобы искать нужный алгоритм было не слишком сложно. (какая то ересь получилась. особенно в конце)

\end{document}